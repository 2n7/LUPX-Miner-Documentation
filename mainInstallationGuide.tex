\documentclass{article}
\usepackage[utf8]{inputenc}
\usepackage[a4paper]{geometry}
\usepackage[english]{babel}
\usepackage{url}
\usepackage{parskip}
\usepackage{graphicx}
\usepackage{amssymb}
\usepackage{enumerate}
\usepackage{amsmath}
\usepackage{verbatim}
\usepackage{hyperref}

\title{LUPX Miner Documentation}
\author{The Lupecoin Project}
\date{January 2019}

\begin{document}

\maketitle

This article is written to clarify the installation process for those who are experiencing problems with the current installers.

\section{Installing the dependencies}
The miner depends on a few modules as listed in the documentation. Three of the four modules are native: "time", "json" and "threading". One module, "web3" is not native.

Since the "web3" module depends on some C++ build-utilities, you'll need to install the "Microsoft Visual Studio 14.0 C++ buildtools" from here: \\ \url{https://visualstudio.microsoft.com/thank-you-downloading-visual-studio/?sku=BuildTools&rel=15}. Note that if you have "Microsoft Visual Studio" already installed, you most probably don't have to install the build-utilities separately! When prompted to install the SDK's or not, do so. \\ 
After you've installed the C++ build-utilities, open one of the installers (.py) (multiple installers are included because not every installer works on every machine). Now, you can open the miner by simply double clicking the "LUPX-Miner.py" file in the "miner" folder. It may take some time before something pops up the first time because the "web3" module has to configure. \\
If no installer seems to work, you can manually install the "web3" module by doing the following steps:
\begin{itemize}
    \item Go to your Python installation path. (Something like: "C/Users/NAME/Appdata/Local/ Programs/Python/PythonXXXX/Scripts"). Here, there will be three different versions of "pip". 
    \item COPY (DO NOT DELETE) the "pip.exe" and place it in the folder of the "installer" of the "LUPX-Miner" folder. So you will have a copy of the "pip.exe" file and the "setup.bat" file in the SAME folder.
    \item Now run the "setup.bat" file. It can happen that there will be some errors in red but that's because three of the four modules are native.
    \item Check if it gives any errors. If it says something like "You need Microsoft Visual C++ 14.0 build-utils", make sure you properly installed those. Here is a thread on issues on the web3 module: \url{https://github.com/ethereum/web3.py/issues/810}
\end{itemize}
\end{document}
